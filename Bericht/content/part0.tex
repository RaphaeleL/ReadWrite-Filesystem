\part{Grundlegende Informationen }
\chapter{Dokumentation}
\section{Aufgabenstellung}
Die Aufgabenstellung bestand darin ein Filesystem auf Fuse Ebene zu Programmieren. Dies soll dazu dienen beliebige Datenträger zu formatieren. Damit kann man beliebige Dateien mit den üblichen Attributen wie Name, Zugriffsrechte und Zugriffsdatum sowie den Zeitstempel. Ebenso soll in dem Filesystem ermöglicht werden, dass wenn eine Datei formatiert wurde, diese in einem freien Platz im Verzeichnis eingetragen werden kann. Die Einbindung wiederrum findet in einem frei wählbaren und leerem Verzeichnis statt, in diesem der Inahlt des Datenträgers erscheinen soll. 

Nicht wie bei üblichen Dateisystemen, bei denen die Daten auf dem Datenträger direkt abgelegt werden, wird in diesem Filesystem mit einer Container Datei gearbeitet. Auf diese wird später detaillierter eingegangen. Außerdem arbeitet MyFS nicht auf Kernel sondern auf Fuse Ebene. Auch hierbei wird später genauerter eingegangen. Somit ist MyFS ein Voirtuelles Dateisystem.

Auf üblichen Dateisystemem, die auf Kernel Ebene arbeiten, finden Anfragen auf dem User Space des VFS statt. Danach der Block- und IO-Layer sowie dem Treiber. Auf MyFS wird diese Vorgangskette vermieden. 

Im zweiten Teil der Aufgabe, also das Read Write Filesystem, war das Ziel mittels verschiedenen Kommandozeilen die Container Datei zu erstellen und es schließlich zu Mounten. Dabei muss die Container Datei sowie die gewünschten Dateien als Parameter übergeben, was wie folgt aussehen kann:
\begin{lstlisting}
./mkfs.myfs container.bin datei1.txt datei2.pdf ... dateiN.mp4
\end{lstlisting}
Um die nun gefüllte Container Datei und das Filesystem zu mounten war folgender Befehl nötig. Hierbei übergibt man die Container Datei, die Log Datei und schließlich das Mount Verzeichnis. 
\begin{lstlisting}
./mount.myfs container.bin log.txt mount}
\end{lstlisting}
Somit ist unser Filesystem mit den N Dateien einsatzbereit. 

Hierbei ist jedoch stark zu differenzieren. Im ersten Aufgabenteil handelt es sich um ein sogenanntes Read Only Filesystem. Das beduetet das lediglich Daten in das Filesystem geschrieben und gelesen werden können. Eine Änderung der Datei ist dabei nicht möglich. Erst im zweiten Aufgabenteil, wird diese Funktion gewährleistet.

\newpage

\section{Vorgaben}
Das Read Only Filesystem hat Regeln die zu beachten sind. Darunter fallen: 
\begin{itemize}
	\item Mindestplatz von 30,1 MB
	\item Maximale Dateinamen Länge von 255 Charakteren
	\item Blockgröße von 512 Byte
	\item Maximale Dateneinträge von 64 Dateien
\end{itemize}
Ebenso ist diverser Code der einige Funktionalitäten bereitstellen sowie ein Testframework gegeben. 

Das Filesystem muss, laut Vorgabe, folgendes erfüllen bzwl folgende Eigenschaften in der Grundstruktur aufweisen:
\begin{itemize}
	\item[\textbf{Superblock}] Informationen zu dem Filesystem (Größe, Position der Einträge, ...)
	\item[\textbf{DMAP}] Verzeichnis der freien Datenblöcke
	\item[\textbf{FAT}] Dateizuordnungstabelle
	\item[\textbf{Root}] Datein im Filesystem mit folgenden Einträgen
	\begin{itemize}
		\item Dateiname
		\item Dateigröße
		\item Benutzer / Gruppe - ID
		\item Zugriffsberechtigungen (mode)
		\item Zeitpunkt letzer Zugriff (atime)
		\item Zeitpunkt letzte Veränderung (mtime)
		\item Zeitpunkt letzte Statusänderung (ctime)
		\item Zeiger auf ersten Datenblock
	\end{itemize}
	\item[\textbf{Daten}] Daten der Datei
\end{itemize} 

Ein Datenblock hat dann folgenden Aufbau / Grundstruktur.

\begin{tabularx}{17cm}{|X|X|X|X|X|}
	\hline
	Superblock & DMAP & FAT & Root & Daten \\
	\hline
\end{tabularx}

\newpage
