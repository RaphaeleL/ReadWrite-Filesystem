	\title{Betriebssysteme Labor \\
		\textit{Hochschule Karlsruhe - Technik und Wirtschaft }\\
		\textit{Erstellung eines Filesystem}
	}
	\author{selu1014, zach1011, atat1011, lira1011}
	\date{\today}
	\maketitle
	\newpage

	\renewcommand{\abstractname}{Zielsetzung}
	\begin{abstract}
		Ziel ist es, ein voll Funktionsfähiges Filesystem zu Programmieren. Hierfür ist das Template für die Aufgaben zum Labor Betriebssysteme. Wenn die notwendige Arbeitsumgegebung eingerichtet wurde, sollte sich das Template-Projekt korrekt übersetzen lassen und dann die Funktionalität des Simple and Stupid File System bereitstelle. 
		\newline
		\noindent \begin{center}Teil 1 - Read Only Filesystem \end{center}
		\noindent \textit{Erstellt werden soll ein Dateisystem MyFS, dass verwendet wird, um Datenträger zu formatieren. Unterstütz werden alle üblichen Datei mit den Attributen wie Name, Größe, Zugriffsrechte etc. Diese Dateien sind hierbei in einem einzigen Verzeichnis angeordnet (also gibt es keine Ordnerstruktur). Eine mit MyFS formatierter Datenträger kann (wie jeder Datenträger mit einem bekannten Dateisystem) in den Verzeichnisbaum eingebunden werden.} 
		\newline
		\noindent \begin{center}Teil 2 - Read Write Filesystem \end{center}
		\noindent \textit{Erstellt werden soll ein Dateisystem MyFS, dass aufbauend auf Teil 1 zusätzlich auch schreiben kann.}
		
	\end{abstract}
	\newpage
	\pagenumbering{arabic} 
	\tableofcontents
	\newpage
	\pagenumbering{arabic} 
	\setcounter{page}{4}
	\setcounter{section}{0}
	\ofoot{\thepage{}}
